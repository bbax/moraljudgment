%COMMENT OUT DOCUMENT HEADER AND BIBLIOGRAPHY TO CREATE FULL DISSERTATION
%\documentclass[11pt]{book}
%\usepackage{multirow}
%\usepackage{harvard}
%\usepackage{setspace}
%\usepackage{subfig}
%\usepackage{geometry}
%\usepackage{fancyhdr}
%\usepackage{amsmath,amsthm,amssymb}
%\usepackage{graphicx}
%\usepackage{hyperref}
%\usepackage{pdflscape}
%\usepackage{fullpage}
%\usepackage{dcolumn}
%\usepackage{epigraph}
%
%\linespread{1.6}
%\graphicspath{ {c:/Users/Public/Figures/} }
%
%\begin{document}

\chapter{The Mediation Model of Moral Judgments of Political Leaders}\label{ch:theory}
%\epigraphhead[10]{\epigraph{``Compromise itself becomes reprehensible, for good men do not go halfway with evil.''}{\emph{Dynamics of the Party System}\\James L. Sundquist}}

Politics is inseparable from the actors involved. We cannot fully understand how citizens perceive the morality of political actions and issues until we incorporate their moral judgments of the actors involved. To this end, I propose a model that incorporates moral judgments of presidential actors into political behavior.

Political psychologists have discovered that most people's political attitudes are not deliberated or reasoned out. \citeasnoun{LodgeTaber2013} argue that most political reasoning is \emph{ex post facto} justification of positions our initial, subconscious impressions nudged us toward. \citeasnoun{Haidt2001} takes a similar position toward moral reasoning.\footnote{One criticism of Haidt's theory is that he focuses mostly on moral judgments regarding sexual purity or other standards of cleanliness, which are less likely to be reasoned through.} Lodge and Taber argue that ``motivated reasoning --- the systematic biasing of judgments in favor of automatically, affectively congruent believes and feelings --- is built into the basic architecture and information processing mechanisms of the brain'' (Location 709 of 7,779). This argument means that citizens are motivated to find justifications for their political predispositions. Republicans will look for reasons that Democrats are wrong because they feel that Democrats are wrong.

Citizens are natural motivated reasoners \cite{LodgeTaber2013}. Most, if not all, of their deliberation and reasoning is focused on justifying conclusions that their unconscious mental processes all ready tend to support.

This chapter develops a theoretical argument for why partisans will be likely to use morally-relevant arguments to justify their attitudes toward presidential candidates and sitting presidents. These moral judgments allow them to connect their political predispositions with their negative assessments of out-party candidates in a way that substantiates those feelings and gives them moral weight. Over time, these moral justifications form a stable moral judgment of the actor, which impacts how individuals react to new information that is relevant to the actor. For example, many Republicans have come to feel that President Obama is not a morally good person who shares their values. This judgment may then influence how they react to other political stimuli and how they engage with politics.

The chapter will also discuss the impacts of moral justifications on political behavior. As described above, moral judgments of presidential actors should lead to connections with other elements of political life. These moral judgments should influence greater party loyalty because they make the in-party more righteous and/or the out-party less so.

\citeasnoun{Groenendyk2013} argues that citizens' have competing motives when it comes to politics. They have a ``side'' that they cheer for and identify with --- their party identification. Individuals are motivated to defend their party, even if their personal opinions diverge from the party's. They also have a desire to be ``good citizens'' whose attitudes are objective and pragmatic (\emph{xv}).


%This chapter covers the book's central theoretical argument: the Mediation Model of Moral Judgments of Political Actors. The argument is not that judgments of actors are decisive on American political life. Partisanship holds the title of undisputed champion of explanatory variables in American politics \cite{Bartels2000,Bartels2002b}. Across partisan camps, the character traits of presidential nominees are not determining elections \cite{Bartels2002}.
%
%My argument is as follows: over the long term, citizens rely on their partisanship for most political judgments. Partisanship is rooted in our social identities, which are largely fixed and unchanged \cite{GreenPalmquistSchickler2002}. However, in the short term individuals' judgments and impressions are more variable and driven through unconscious processes \cite{LodgeTaber2013,Todorovetal2005}. Over the medium term, however, presidential actors provide a prominent context for political information [CITATION?].
%
%When an actor first emerges on the national stage, citizens lack schema for judging the actor. Thus initially they rely on partisan cues. These partisan cues bias them toward making a negative judgment of out-party actors \cite{Goren2007}.



%
%When it comes to issues and policies, the public has attitudes not preferences \cite{Bartels2003}.
%
%The public does have preferences for parties over the long term and for candidates over the short and medium term.
%
%These preferences are quite strong among some citizens, but not for others. Strong partisans have strong stable preferences; leaners have weaker preferences, even if their votes at the ballot box ultimately mirrors that of stronger partisans.
%
%Over the short- and medium- term, much of the political information that individuals encounter is framed through discussion of presidential actors \cite{McAllister2007}.
%
%Actors are moral defined through their policy positions than in past decades \cite{Gilensetal2007}, but those policy positions have moral implications \cite{Kolevaetal2012} which in turn have implications for judgments of character \cite{Clifford2014}.
%




\section{Models of Morality and Politics}
In the past ten to fifteen years, experimental philosophers and social psychologists have overthrown their disciplines' long-standing theory of morals as products of conscious, deliberative reasoning. Previously, scholars understood morals as rational social constructs that humans learn as they grow up \cite{Kohlberg1981}. These morals were supposed to be based on conscious reasoning that could be justified and explained. However, researchers experimenting with moral intuitions discovered that emotions often explain morality better \cite{Cushmanetal2006,Haidt2001}. Researchers found that people often could not give justifications for their moral judgments \cite{Haidtetal1993}. People do not reason through moral problems; they \emph{feel} their way through them. External influences that affect our emotional dispositions, such as fart sprays \cite{Schnalletal2008}, unpleasant tastes \cite{Eskineetal2011}, and the proximity of hand sanitizer \cite{HelzerPizarro2011}, can have significant impacts on how we make moral judgments. Something as seemingly innocuous as emphasizing cleanliness can make people more conservative in their judgments \cite{ZhongLiljenquist2006}.

Political scientists building on these developments have largely interpreted the findings to mean that emotions drive political morality \cite{Haidt2001}, while deliberate moral reasoning is largely irrelevant \cite{CliffordJerit2013,Kolevaetal2012,WeberFederico2012}. However, neurologists, psychologists, and experimental philosophers debate whether it dismisses reasoning too harshly. There is some evidence that different contexts lead to different modes of moral reasoning \cite{CushmanYoungGreene2010,Greeneetal2001}. This debate has significant implications for politics because it is often thought to rely on moral reasoning and persuasion. Understanding when different styles are more likely will help understand how citizens use morality in their political behavior.

%These emerging theories of morals and politics are a resurrection, in a sense, of the revisionist challenges \cite{Achen1992,Fiorina1981} to the modified ``Michigan'' model of political identity \cite{CarseyLayman2006,GreenPalmquistSchickler2002}. The theories argue that partisanship and ideology are the results of individuals' moral values.

%The ``trolley problem'' illustrates the debate between moral emotions and moral reasoning.\footnote{Scholars in social psychology, neuroscience, experimental philosophy, and other fields have found the trolley problem a useful paradigm for exploring moral intuitions. See \citeasnoun{Edmonds2013} for a comprehensive and accessible overview of the ``trolley problem'' literature.} The moral philosopher Philippa Foot \citeasnoun{Foot1967} originally proposed the trolley problem as a thought experiment for exploring the complex moral arguments surrounding abortion. Since its introduction, scholars have introduced numerous mutations and variations of the trolley problem. Recently researchers in psychology and experimental philosophy have put the trolley problem to the test and found some paradoxes in how humans think about moral decisions.

%The trolley problem has two basic forms. The first is sometimes called the ``lever'' or ``bystander'' set-up. In this version, an out-of-control trolley is racing down its tracks toward five people who cannot get out of the way. Nearby is a switch or lever that can redirect the trolley onto a secondary track, thus saving the lives of the five. However, there is another person on the spur who will be killed. Most people believe it is morally acceptable to pull the lever. In a second variant proposed by \citeasnoun{Thomson1985}, often called ``footbridge,''\footnote{This situation is often referred to as ``Fat Man,'' which seems to be a more common name, but I use ``footbridge'' to avoid emphasizing the man's weight.} there is a footbridge over the track instead of a secondary track. A man is positioned directly above the tracks on the bridge in perfect position to stop the trolley before it hits the five people in its path---if another actor pushes him into the trolley's path and to his death. Most people say it is not moral to push the man from the footbridge \cite{Cushmanetal2006,Greeneetal2001,Hauseretal2007}.\footnote{There are modifications that narrow the difference between how people react to these situations, but none completely overcomes the disparity \cite{Edmonds2013}.}

%The results of experiments using the trolley problem tend to support a dual-process model of moral judgment. \citeasnoun{Greeneetal2001} argue that people reason differently about these two situations because of a difference in how \emph{personal} the acts are. In the lever set-up, the action is distant and impersonal, but it is personal and intimate in footbridge. Their costs and benefits are the same: one life to save five. Yet people bring very different moral considerations to bear on each. In lever, people tend to make the consequentialist decision because the personal distance allows them to more cooly reason through the consequences. In footbridge, people react emotionally to the intimate act of pushing a man to his death; these emotions overwhelm consequentialist reasoning and make their judges deontological. The emotion makes them unwilling to do something they feel is wrong, even though it would save five lives. People make different moral judgments in these situations because they \emph{feel} them differently.

%The set-ups used in the trolley literature are simplifications; nevertheless, they provide new insights into understanding how citizens might use their moral intuitions to make judgments regarding politics and governance. Emotions are central to moral judgment, but the context of a judgment can influence moral decisions. In some circumstances judgments are dependent on strong emotional intuitions, while in others people reason their way to a decision.\footnote{A full exploration of the reasons for the differences reactions are beyond this discussion, but there is some opposition to the conclusion that people use moral reasoning in some situations and moral emotions in others. See, e.g., \citeasnoun{Kahaneetal2012} and \citeasnoun{Prinz2007}.} This finding mirrors how citizens tend to think about politics. In some circumstances, they think harder and exert more cognitive effort \cite{Petersenetal2013}; in others, they rely on emotional intuitions \cite{Haidt2012}. Again, in some circumstances, people are motivated to construct accurate and informed opinions, while in others they are motivated to construct opinions that justify their predispositions \cite{TaberLodgeGlathar2001}. I believe an important determinant of which process is use is party identification. When judging the other party, judgments are quick and emotion. When judging one's own party, individuals are more likely to engage in more cognitive reasoning. We \emph{feel} our judgments against the other side; we \emph{think} our judgments of our side.





\section{Mediation Model of Moral Judgments of Political Actors}\label{sec:theory_model}
Attitudes toward actors are not independent of other important political attitudes, including moral intuitions and moral convictions. I argue that moral judgments of actors work as a mediation between these attitudes and political behaviors. I propose a mediation theory of moral judgments of political actors.\footnote{These actors could be political parties or major candidates. For the purposes here, I focus on political candidates and leave aside moral feelings toward the parties.} This model relies on the theories of hot cognition and on-line information processing.

[[Update figure with simple one used for Pew job talk]] Figure \ref{fig:theory_mediation_model} graphically summarizes the model I propose. Citizens have a number of predispositions, such as moral intuitions, moral convictions, political identities, and ideological attitudes, that impact their political behavior (and attitudes). The strength and importance of these predispositions will vary across individuals \cite{Zaller1992}. When these nodes are activated and brought into active memory, their relationship with the actors involved will be activated as well \cite{TaberLodgeGlathar2001}. The predispositions have a direct effect on political behavior (dashed lines), and they have an effect on moral judgments of actors (solid lines). Repeated activations strengthen these connections between the actors and the other morally charged nodes. Cognitive judgments and affective reactions will transfer from the left-hand nodes to behavior and to judgments of the actors (See chapter 5 of \citeasnoun{LodgeTaber2013}). Once developed, moral judgments of political actors will gain in primacy as the reasons for the judgments fade. The thick black arrow from moral judgments to behavior increases its significance and is activated when individuals encounter stories in the media emphasizing the actors and other stimuli.

\begin{figure}[htb]
  \centering
  \caption[The Mediation Model of Moral Judgments of Political Actors]{\textbf{The Mediation Model of Moral Judgments of Political Actors}\\ \footnotesize \textit{Note}: This figure shows a simplified model of how moral judgments of political actors mediate other moral attitudes' and identities relationship with political behavior. Moral conviction, moral intuitions, social identities, and ideological attitudes have direct effects on political behavior that have been well-documented, but I theorize they also have effects mediated through moral judgments of actors. An on-line model of information processing leads partisans to encounter morally relevant information, those encounters leave affective judgments of actors, the reasons are forgotten, and the affectively-charged moral judgments of the actors remain to influence behavior.}\label{fig:theory_mediation_model}
  {\includegraphics[width=0.9\textwidth]{Mediation_Model_Simple.pdf}} \\
\end{figure}

People construct their political attitudes through process of motivated reasoning \cite{TaberLodgeGlathar2001}, and actors are an important part of that construction. A multitude of nodes activate when individuals encounter information regarding prominent political entities. Media coverage increasingly focuses on individual actors \cite{McAllister2007}, instead of the parties, which suggests that a common form of activation is hearing about the actors involved. The personalities of politics dominate media coverage, so stories dealing with any of the important predispositions are also likely to mention the actors involved (e.g. President Obama supports same-sex marriage; President Bush passes tax cuts). The web of connections between actors and other issues form the schemata that individuals rely on to make political judgments.

Moral judgments are the elements of the schemata of actors that are values-based and have become morally charged through affective weighting \cite{TaberLodgeGlathar2001}. Initially, citizens do not have schemata for candidates and elected officials. There are partial schemata in their minds constructed from partisan considerations, initial impressions, etc., but they are not crystalized. These schemata fill in as citizens encounter information regarding the actors that activates moral intuitions. These reactions are stored in longer-term memory through the affective running tally. The reasons for these judgments may be forgotten, but citizens will continue to believe actors have certain traits and those traits will activate affective reactions in their minds.

Moral intuitions are nodes in long-term memory that are so powerfully felt that their accompanying ``affective tags'' transfer to related nodes that are attached to those intuitions \cite{Haidt2001,Haidt2012}. Furthermore, moral intuitions will have greater ``node strength'' which means they are more likely to be activated from long-term memory. The emotions they activate will influence the information and considerations that are moved into ``working memory'' \cite[202-203]{TaberLodgeGlathar2001}. This relationship means that objects or people that activate a moral intuition will influence how related groups, concepts, and individuals are perceived. Political persons have strong evaluative tallies attached to them, and these ``feelings become information'' \cite[456]{LodgeTaber2005}, especially when they are morally relevant tallies.

Consider an example of the nodes activated when a Republican thinks about President Obama. The schemata of President Obama is likely well developed in the Republican's mind, so there are a number of clearly connected nodes. I hypothesize that when this Republican thinks about President Obama it will activate two morally important nodes. The first is morally relevant emotions, especially aversion, that have transferred to the president through affect transfer \cite{LodgeTaber2013}. The second node will be a global judgment of whether President Obama is a morally good or morally bad entity (as a person, a president, a politician, a father, etc., etc.). These two nodes will be channels through which other connections are interpreted. There are other linked nodes, but these two constructs will have particular influence because they form the basic judgments of President Obama. When new decisions and attitudes appear on the political landscape, people associate them with the moral emotions and judgments they have formed regarding the actor. In this way, an idea originating in a conservative think tank---the individual mandate---becomes an intolerable assault on liberty when an unpopular out-party president adopts it.\footnote{Overviews of the history of the individual mandate are available at \url{http://www.healthreformwatch.com/2011/02/14/the-individual-mandate-a-brief-history-part-i-conservative-origins/} and \url{http://healthcarereform.procon.org/view.resource.php?resourceID=004182}.} The negative judgments of President Obama biased subsequent judgments of the mandate. The individual mandate was uncontroversial as a conservative proposal in 1990s think-tank circles, but it quickly became an ``idea \emph{non grata}'' when it became a central piece of President Obama's health reform efforts.\footnote{The dynamics of support for the individual mandate are a bit more complicated than just Republicans turning against it. A poll in 2008 found Republicans and Democrats similarly opposed to the mandate with 50\% strongly opposed. Two years later in late 2010, each group had shifted by about 30 percentage points. Republicans now opposed the mandate at about 80\%, while Democrats now supported it at about 45\% where they only supported it at 10\% in 2008. Sources: Henry J. Kaiser Family Foundation, Harvard School of Public Health, National Public Radio. ``The Public on Requiring Individuals to Have Health Insurance Survey, February 14 - February 24, 2008. Fox News and Opinion Dynamics, "Fox News Poll," Dec. 14-15, 2010. Accessed through Roper's iPoll database and at \url{http://healthcarereform.procon.org/view.resource.php?resourceID=004168}.}

The judgments of moral goodness and the moral emotions attached to actors are how I conceptualize moral judgments. This approach combines cognitive character judgments with affective reactions. \citeasnoun{TaberLodgeGlathar2001} argue that scholars ``have established much too strong a dichotomy between affect and cognition'' (198). Moral judgments of actors are a combination of the two. They include cognitively constructed character judgments together with the morally relevant emotion of aversion. Trait judgments fall into two primary classes: morality and competence.\footnote{Scholars have identified a bewildering number of trait dimensions for judging actors \cite{BenoitMcHale2004,Funk1996b}. I rely on two: morality and competence. More on this distinction is below.} Affect has two primary dimensions: positive and negative \cite{HuddyFeldmanCassese2007,Marcusetal2000}, but negative emotions have two subdimensions of anxiety and aversion. Aversion is the moral emotion because of its high pro-social tendency (meaning it serves to uphold the social order) and its disinterested elicitors (meaning we feel it for others, not just ourselves) \cite{Haidt2003,Petersen2010,SteenbergenEllis2006}.

I define a moral judgment of a political actor as an affectively-charged negative evaluation of an actor's values-based characteristics.\footnote{This definition follows Haidt's \citeyear{Haidt2001} definition of moral judgments as ``evaluations (good vs. bad) of the actions or character of a person that are made with respect to a set of virtues held to be obligatory by a culture or subculture'' (817).} This definition excludes party identification, ideology, specific policy positions, or traits such as ``competent'' or ``intelligent'' as the explicit grounds for the judgment---though party identification and other factors will certainly condition how and when these judgments are made. The judgment may be based on information coming from issue positions or ideology, but the effect of moral judgment remains even when controlling on these factors.

Traits are value-based when they share the characteristics of values \cite[20]{Schwartz1994}, meaning they ``are (1) abstract beliefs (2) about desirable end states or behaviors that (3) transcend specific situations, (4) guide evaluation and behavior, and (5) can be rank-ordered in terms of relative importance'' \cite[882]{Goren2005}. Integrity, for example, is abstract, reflects a desirable behavior, transcends narrows contexts, guides evaluation, and can be ranked against other desirable characteristics. On the other hand, knowledgeable shares some of these characteristics, but not all. Competence is less abstract; there are clearer objective standards for its definition and measurement. It is not just desirable but necessary. Knowledgeable is highly dependent on specific situations. A knowledgeable president will be very different from a knowledgeable doctor or janitor. Knowledgeable does share the last two traits. Transcending specific situations and abstract beliefs are the two dimensions where moral and competence traits most diverge.

I hypothesize that these moral judgments rest primarily on the traits connected with an actors' \emph{suitability} for political office. Non-moral judgments focus on actors' \emph{ability} or competence to hold office. These values-based judgments of actors' integrity and empathy interact with the moral emotion of aversion. I focus on integrity and empathy because they are closely related to Haidt et al's \cite{Haidt2012,Grahametal2012} moral foundations theory. Aversion is the relevant emotion because previous studies have found it is closely connected with moral conviction \cite{Petersen2010,SkitkaWisneski2011}and moral outrage \cite{SteenbergenEllis2006}. Other character traits such as intelligence, competence, and leadership are not necessarily values-based, so I exclude them as indicators of moral judgments, though they are important controls for non-moral attitudes toward politicians.

My definition of moral judgments relies on a distinction between moral traits that are connected with values and suitability for holding office and competence traits that are connected with ability for holding office. Scholars have previously made this distinction \cite{Funk1996b,Kinder1986}, though typically scholars disaggregate traits into integrity, empathy, leadership, and competence \cite{Goren2007,Hayes2005}. Dividing trait judgments into these classes follows the dichotomy that \citeasnoun{CuddyFiskeGlick2007} identify in person perception. Our perceptions of other people and groups first assess their competence---whether they can harm us---and second we assess their intentions, warmth, or morality---whether they will harm us. In the European context, researchers have found that morality traits as the strongest and most influential traits for attitudes toward politicians \cite{Wojciszke2005,WojciszkeKlusek1996}.





\section{Partisanship and Morals}
Voters are not disinterested evaluators solely interested in supporting the party or candidate that most closely agrees with their preferences. Instead, most voters are deeply committed to a party and process information through a ``perceptual screen'' that favors that party \cite{Campbelletal1960}. Partisanship influences nearly every aspect of political behavior and attitudes, but the evidence is mixed on the relationship between attitudes and party identification. There is evidence that partisans adjust their core moral values to match those of their party \cite{Goren2005}. There is also evidence that people sort into the party that best fits their values \cite{Levendusky2009a}. And there is evidence that people do both \cite{HightonKam2011}. Nevertheless, there is significant evidence that partisanship largely explains who wins elections. Issues and personal characteristics seem to have very little influence \cite{Bartels2002}.

Some people believe that elections are contests of principles and policy; yet, often they are more atavistic battles between political tribes. There is little evidence that political principles divide these tribes \cite{DiMaggioetal1996,FiorinaAbramsPope2005,FiorinaAbrams2008}, though there is an increasing proportion that does take consistent issues-based positions.\footnote{The Pew Research Center found in the spring of 2014 that 20\% of Republicans take consistently conservative positions, up from 13\% in 1994. Pew found that 23\% of Democrats take consistently liberal positions. Only 5\% of Democrats were consistently liberal in 1994 } There is little evidence of issue voting when it is put to a strict test \cite{MeierCampbell1979}. However, when issue attitudes are corrected for measurement error, the cumulative effect of individuals' positions is more significant, together rivaling the impact of party identification \cite{Ansolabehereetal2008}. \citeasnoun{Lenz2012} finds that as people learn the positions of their preferred candidates and parties, people follow. Instead of seeking a different candidate that is closer to their opinions, people change their opinions to match those of the actors they support. The balance of evidence suggests that issue voting does occur, but it is difficult to separate its effects from partisanship. Its influence is also typically weaker than party identification \cite{Jacoby2010,LewisBecketal2008}. Nevertheless, identity \cite{Levendusky2009a,Mason2013,MasonForthcoming} and moral intuitions \cite{Haidt2012} divide the parties more than political principles.

The two major U.S. political parties represent diverse groups of people, and partisans are sometimes united more by what they oppose than what they support. If partisans feel conflict between their identity and their attitudes, they ``[call] to mind negative attitudes and stereotypes of the opposition party'' to protect against the threat to their identity \cite[464]{Groenendyk2012}. This ``lesser of two evils'' justification implies that partisans focus on the evil of the other party, thus reinforcing negative moral judgments. Thus, partisanship can be thought of as a moral intuition in and of itself in some cases. For instance, political scientists once theorized that citizens keep retrospective running tallies of the parties' performance in their heads. They were supposed to use this tally to evaluate objectively which party best served their interests. If this were true, public opinion surveys should show a gradual convergence of opinion among Republicans and Democrats as they both kept accurate, unbiased tallies. In fact, there is no convergence \cite{Bartels2002b}. Partisans stay true to their party through thick and thin; it is a value they appear unwilling to compromise on. Though it is not immediately clear if the partisan differences are the result of staying true to their parties or if they result from cognitive limitations that prevent many people from accurately identifying the parties' performances. Voters are myopic and often emphasize the wrong information when assessing political information \cite{Bartels2008}.

Moral judgments are likely to be highly partisan because more often than not motivational goals drive how individuals construct their political opinions \cite{TaberLodgeGlathar2001}. Even citizens who lean toward one party are motivated to interpret information in ways that justify their partisan predisposition. When people have an emotional stake in an issue, their reasoning is different and more emotional than when they are indifferent \cite{Westenetal2006}. Furthermore, giving issues partisan labels makes committed partisans engage in more effortful thinking to justify their party's position \cite{Petersenetal2013}. Partisans think harder when they have to defend their own, but they appear to rely on automatic, affective judgments of the other side, which are similar to moral intuitions.

All of these factors combine to make it highly likely that partisanship leads individuals to see the other side in moral terms. Moral intuitions evolved to bin groups together \cite{Haidt2012}. These intuitions can contribute to moral exclusion and vilification of others outside the group. Political scientists mostly discuss moral exclusion through its most extreme cases of genocide or xenophobia (see, e.g., \citeasnoun{Staub2000}), yet a more subtle form of moral exclusion appears to be occurring between partisans. For example, \citeasnoun{Iyengaretal2012} find that today partisans are less willing to accept a child marrying a member of the other party than a few decades ago. During the 2008 election, partisans who perceived greater social divisions between the parties were also more likely to endorse smears about the candidates (e.g. Obama is a Muslim; McCain is senile) \cite{Kosloffetal2010}. This subtle exclusion is not leading to violence, but it is increasing the affective and behavioral polarization between partisans \cite{Iyengaretal2012,MasonForthcoming}. I believe a mechanism for affective polarization is moral judgments of actors. Both partisan sorting and negative campaign rhetoric would contribute to increased feelings that the other party's leaders are morally bad because each activates the moral community of partisanship and emphasizes the negative aspects of the other party.

Partisanship is not the only consideration that people bring to bear when making judgments, but I theorize it is at least as important as the content of moral issues. When given enough information about policies, partisans rely about equally on issues and partisan labels \cite{Bullock2011}. When information about a judgment target is low, then the importance of partisan cues increases \cite{Cohen2003,Kam2005}. Because they often emphasize partisanship over individual issues and policies, people are also likely to have attitudes about the morality of actors at least as much as they will have attitudes about the morality of issues and policies.

\citeasnoun{Stoker1993} finds that partisanship significantly moderated the reactions of Democrats who endorse traditional moral values. When Senator Gary Hart's affair with Donna Rice was exposed in May 1987, the opinions of Hart among Republicans with traditional moral values became significantly more negative. Among Democrats, however, moral values did not have a significant negative effect. At the very least, this evidence shows that partisanship can cause people to overcome their moral intuitions. Moral Foundations Theory implies that morally traditional Democrats and Republicans both emphasize the purity moral intuition; therefore, information relevant to that intuition should have a similar effect among both groups. Stoker finds that it did not when it came to Senator Hart's scandal.


\section{Moral Issues \emph{and} Moral Actors}
The nature of moral judgments and the influence of partisanship suggest that issues and actors will have complimentary roles when it comes to moral judgments. Just as issue content and partisan cues compliment information processing, I argue that judgments of actors and judgments of issues are complimentary. Often political scientists point to morality as an example of how citizens organize politics according to principles, instead of social identity. As a result, most of the scholarly literature on morals and politics focuses on issues and policies. Even recent work building on new theories from social psychology have focused on issues. However, there is good reason to believe that moral judgments of actors have an equally significant influence.

As discussed in the previous section, there is limited evidence to support the idea that individuals vote according to their issue preferences; instead, they appear to vote according to their partisan identity. Many individuals lack of information to vote their attitudes \cite{DelliCarpiniKeeter1996}, and even when relying on shortcuts and cues voters are often unable to vote as if they were better informed \cite{Bartels1996}. The lack of information and reliable shortcuts contributes to most Americans' lacking true preferences on most issues. Though the public's attitudes are more stable than \citeasnoun{Converse1964} argued (see, e.g. \citeasnoun{Ansolabehereetal2008}), there is good evidence that most Americans do not carry around real preferences on many political issues. Citizens ``have `meaningful beliefs' but those beliefs are not sufficiently complete and coherent to serve as a satisfactory starting point for democratic theory, at least as it is conventionally understood'' \cite[48-49]{Bartels2003}. Instead, they have ``attitudes,'' which are a tendency to favor or disfavor a specific entity \cite[52]{Bartels2003}. Attitudes are not stable or consistent like preferences; they are tendencies that are variable and imprecise. This inconsistency calls into doubt the assertion that citizens can easily and consistently call on stable moral preferences \cite{SkitkaMorgan2014} or respond the same way each time the sample prime activates a moral intuition. It is likely that some individuals on some issues carry around stable moral preferences; however, it is more likely that citizens carry around stable moral judgments of actors that are informed by their moral intuitions and moral convictions. Issues come and go on the national agenda (though some make repeated visits), but in the short-term actors at the presidential level are fairly stable. In the long-term, attitudes toward the parties change very slowly, if at all \cite{GreenPalmquistSchickler2002,JenningsNiemi1981}.

Some citizens do hold attitudes that are ``grounded in core beliefs about fundamental right and wrong'' \cite[96]{SkitkaMorgan2014}, but as \citeasnoun{Fiorina1981} put it, ``[voters pass] judgment on leaders, not policies" (11). By this he meant that when citizens engage in politics they judge actors for the most part, not issues or policies. In fact, it's likely that more people develop moral convictions that relate to actors than issue-based moral conviction because more people possess the resources and attitudes to judge actors.

People forget most of the political information they encounter, yet the information leaves an enduring mark on their evaluations \cite{LodgeMcGrawStroh1989,LodgeSteenbergenBrau1995}. The specifics of the information may be lost, but it still leaves an affective mark in people's minds. People have running affective tallies that indicate how they feel toward political actors. Information encountered first has the most impact on these tallies. These bits of information interact with people's moral intuitions to shape the affective perceptions of candidates and elected officials. For example, an individual with strong moral intuitions regarding fairness will attach similarly strong ``affective tags'' to actors when they encounter information relevant to that intuition. Learning a presidential candidate does not support gay marriage would transfer the negative affect that the fairness intuition generates onto the actor. It is likely that the individual will forget the cause of the affective tag, but they will have transferred the emotion of the intuition to the actor. With repeated activations and transfers over time, I hypothesize that attitudes toward the actors will crystalize into moral judgments of those actors. Those judgments will have their own impacts, independent of the separate moral intuitions, that significantly affect political behavior.

Psychologists have found that political actors have moral meaning. For example, \citeasnoun{Effronetal2009} found that white respondents who expressed support for President Obama then felt they had the moral credentials to favor their race over African Americans. Similarly, other scholars found that President Obama's election in 2008 immediately made whites less supportive of efforts to ameliorate racial disparities \cite{Kaiseretal2009}. A black president, in a sense, absolved whites of their responsibilities on the moral question of racial justice. Similarly, subconscious primes of attitudes toward politicians can significantly impact political behavior \cite{WeinbergerWesten2008}.

Just as political scientists have updated their models of political learning to better incorporate policy cues together with partisan cues \cite{Bullock2011,Highton2012}, I believe that theories of morals in politics need updating to incorporate the role of actors to compliment the strong findings regarding moral conviction and moral intuitions.

[[Discuss: moral emotions focus on individuals (Peterson); moral judgments focus on individuals' actions (Clifford et al 2015)]]


\section{Conceptualizing Moral Judgments}
My conceptualization of moral judgments combines cognitive character judgments with affective reactions. \citeasnoun{TaberLodgeGlathar2001} argue that scholars ``have established much too strong a dichotomy between affect and cognition'' (198). Moral judgments of actors are a combination of the two. They include cognitively constructed character judgments together with the morally relevant emotion of aversion.

I define a moral judgment of a political actor as an affectively-charged negative evaluation of an actor's values-based characteristics.\footnote{This definition follows Haidt's \citeyear{Haidt2001} definition of moral judgments: ``evaluations (good vs. bad) of the actions or character of a person that are made with respect to a set of virtues held to be obligatory by a culture or subculture'' (817).} This definition excludes party identification, ideology, specific policy positions, or traits such as ``competent'' or ``intelligent'' as the explicit grounds for the judgment---though party identification and other factors will certainly condition how and when these judgments are made. The judgment may be based on information coming from issue positions or ideology, but the effect of moral judgment remains even when controlling on these factors. I hypothesize that these moral judgments rest primarily on the traits connected with an actors' \emph{suitability} for political office. Non-moral judgments focus on actors' \emph{ability} or competence to hold office. These values-based judgments of actors' integrity and empathy interact with the moral emotion of aversion. I focus on integrity and empathy because they are closely related to Haidt et al's \cite{Haidt2012,Grahametal2012} moral foundations theory. Aversion is the relevant emotion because previous studies have found it is closely connected with moral conviction \cite{Petersen2010,SkitkaWisneski2011}and moral outrage \cite{SteenbergenEllis2006}. Other character traits such as intelligence, competence, and leadership are not necessarily values-based, so I exclude them as indicators of moral judgments, though they are important controls for non-moral attitudes toward politicians.

Moral judgments are the elements of the schemata of actors that are values-based and have become morally charged through affective weighting \cite{TaberLodgeGlathar2001}. Initially, citizens do not have schemata for candidates and elected officials. There are partial schemata in their minds constructed from partisan considerations, initial impressions, etc., but they are not crystalized. These schemata fill in as citizens encounter information regarding the actors that activates moral intuitions. These reactions are stored in longer-term memory through the affective running tally. The reasons for these judgments may be forgotten, but citizens will continue to believe actors have certain traits and those traits will activate affective reactions in their minds.

Moral intuitions are nodes in long-term memory that are so powerfully felt that their accompanying ``affective tags'' transfer to related nodes that are attached to those intuitions \cite{Haidt2001,Haidt2012}. Furthermore, moral intuitions will have greater ``node strength'' which means they are more likely to be activated from long-term memory. The emotions they activate will influence the information and considerations that are moved into ``working memory'' \cite[202-203]{TaberLodgeGlathar2001}. This relationship means that objects or people that activate a moral intuition will influence how related groups, concepts, and individuals are perceived. Political leaders appeal different moral intuitions in order to appeal to their respective ideological bases \cite{CliffordJerit2013,Grahametal2009}, so at the same time that actors try to appeal to their supporters they may alienate members of the opposition who do not share those more intuitions.

The literature on perceptions of political actors has largely kept emotions and traits separate. Recent developments in the traits literature have found a connection between moral intuitions and how people judge politicians. \citeasnoun{Clifford2014} finds that elected officials' policy positions signal to citizens the actors' moral character traits, especially for individuals who emphasize moral intuitions related to the policies. For example, Democrats who emphasize caring are more positive on that trait when they learn a Republican politician opposes the death penalty. Clifford argues that moral foundations shape trait attributions because traits associated with favored moral intuitions are the most accessible. Overall, the evidence suggests that trait judgments are a mixture of policy positions \cite{Clifford2014}, partisan stereotypes \cite{Hayes2005,Hayes2011}, personality \cite{Kinder1986}, and behavior \cite{Funk1996a}. There is evidence that the importance of traits varies as campaigns and elite messages emphasize different aspects of candidates' personalities \cite{Funk1999}.

\subsection{Judgments of Politicians' Character Traits}
Scholar have identified a number of dimensions along which people make character trait judgments. A number of different typologies have been proposed [[CITATIONS FROM BENOIT CHAPTER IN HACKER 2005]], but perhaps the most common is to divide traits between integrity, empathy, leadership, competence \cite{Goren2007}.



\subsection{Affect: Aversion vs. Anxiety}

The literature on the role of affect in politics has focused largely on anxiety \cite{Marcusetal2000}, but scholars have turned their attention to aversion in recent years. Aversion is related to core moral values and is activated by affronts to those values \cite{SteenbergenEllis2006}, but it has not been connected with specific trait judgments. Anger "regulates behavior and opinions towards people" \cite[358]{Petersen2010} because of its strong pro-social tendencies and connection to disinterested elicitors \cite[854]{Haidt2003}.

There is evidence that individuals will often conflate anxiety and aversion into a single negative affective state \cite{HuddyFeldmanCassese2007,Petersen2010}. Often it is impossible to distinguish the two \cite{Marcusetal2006}.  \citeasnoun{HuddyFeldmanCassese2007} find that in the lead-up to the Iraq War emotions toward President George W. Bush collapsed to a single dimension. Anxiety and aversion were indistinguishable within a unified negative emotional reaction. However, it is possible that the data supports this finding because the survey only asked about a single dimension of President Bush. Political actors, especially the President of the United States, are not unidimensional attitude objects. The president is many things from the leader of his or her party to the commander-in-chief of the armed forces. Each role that the president fulfills could bring its own set of emotional reactions. In the same way that how negative and positive emotions are often weakly correlated because people can feel multiple emotions regarding someone or something simultaneously \cite{Watsonetal1999}, it is possible that people have multiple emotions about multiple aspects of political actors. For example, after 9-11 President George W. Bush offered offered comfort and purpose to many Americans. His approval ratings were extremely high for a few weeks after the attacks, which suggests positive feelings among Republicans and Democrats. It is possible these positive feelings of pride or hope may remain attached to one element of Bush's role as president, while negative emotions that came later regarding President Bush's integrity before the Iraq War may attach to a separate part of people's conceptualizations of the president.

I hypothesize that negative judgments of a politician's moral character traits will elicit emotions related to aversion. Scholars have identified aversion as the emotion most closely connected with morality \cite{Petersen2010,SteenbergenEllis2006}, so I expect it to be present when individuals judge that a politician lacks a core moral attribute such as integrity or empathy. Furthermore, I hypothesize that the aversion will be attached to the trait judgment itself and is not just a universal feeling toward the actor.




\section{Theory}
In this section, I will explain why moral attitudes toward actors matter. Then I will lay out a mediation model of moral judgments of actors. This model informs the observational analysis in the rest of the chapter. In Section \ref{sec:hypotheses}, I elaborate on the connection between moral judgments and the three classes of dependent variable that form the analysis before moving onto a discussion of the data and the analysis.

\subsection{Moral Judgments of Issues \emph{and} Actors}
Recently, two theories from social psychology have reshaped how political scientists understand the relationship between morals and politics. Perhaps the most influential theory is Haidt et al's Moral Foundations Theory (MFT) \cite{Grahametal2009,Haidt2012}. The second is Skitka's Moral Conviction Theory \cite{Skitka2010}. These two theories are complimentary because they describe different stages in moral judgments on political issues. MFT argues that morals are largely intuitive \cite{Haidt2001}. People have various moral intuitions that automatically activate when they encounter relevant stimuli. These intuitions nudge people toward their favored conclusion, and moral reasoning occurs \emph{post hoc} to defend and justify conclusions that were originally felt not thought. Moral intuitions differ between liberals and conservatives, so Haidt et al argue these differences explain why the ideological groups disagree so ardently. Recent research has found support for this argument. Moral intuitions are highly predictive of issue positions on a host of issues that divide liberals and conservatives \cite{Kolevaetal2012}.\footnote{Scholars working within the MFT framework argue that moral intuitions explain more than ideology or partisanship. This may be the case, but current studies use 20-30 questions to measure intuitions and one or two to measure the other constructs. More questions will nearly always lead to greater validity and stronger relationships.}

Moral conviction theory holds that people have ``meta-cognitions'' that lead them to recognize that some of their attitudes are ``grounded in core beliefs about fundamental right and wrong'' \cite[96]{SkitkaMorgan2014}. Moral convictions lead people to recognize that the issues are moral, and thus they are much less likely to compromise and have harsher attitudes toward the other side \cite{SkitkaWisneski2011}. Researchers have found that moral convictions extend to issues well beyond the traditional domain of ``morality politics'' \cite{Ryan2014}. Previously political scientists thought the issues of sex and sin were a separate domain from other political considerations \cite{Mooney2001}. This assumption was faulty because moral convictions can form around any issue that people believe is ``right or wrong, moral or immoral'' \cite{SkitkaBauman2008}.

There is strong empirical evidence supporting both theories, but they are also missing part of the political landscape by emphasizing issues and not considering actors. Both theories are susceptible to criticisms because they rely on a mechanism that is based on political issues. The theories imply that people have deontological moral principles regarding political issues, but few people have very stable issue preferences that would suggest they carry around such principles \cite{Bartels2003}. Stable issue preferences are largely absent from most citizens' minds; relatively few people have them. Instead, citizens have attitudes, which are ``a psychological tendency [to evaluate] a particular entity with some degree of favor or disfavor'' \cite[52]{Bartels2003}. This criticism is most pointed for moral conviction theory, but it can apply to moral foundations as well because the stable moral intuitions that Haidt et al theorize imply that stable issue preferences should develop. These stable preferences are prevalent among the well-informed and political interested. A great deal of Haidt's data rely on survey data using volunteer samples of respondents who sought out his website; these samples probably overstate the moral sophistication and polarization of the electorate. It is hard to see how moral intuitions can drive political life when so few members of the public seem to have the stable, consistent preferences these intuitions should create.

Another criticism is these theories largely ignore the social and institutional aspects of politics. Most Americans are not disinterested observers willing to consider both sides of an argument. Parties nominate candidates and have leaders who actively shape the political landscape. There is limited evidence that Americans will follow their moral intuitions or convictions where they lead. Instead, most Americans are motivated to support their chosen side \cite{TaberLodgeGlathar2001}. Partisans are motivated reasoners when it comes to policies, in particular. \citeasnoun{Lenz2012} finds strong evidence that citizens follow their party's leaders on most issues. The issues he examines are not explicitly moral, but they have moral implications.\footnote{Though Lenz's issues are not explicitly moral, if moral conviction is correct there should be more evidence of partisans resisting their party because the issue is non-negotiable and not every party position can align with the entire party's moral intuitions and convictions. Lenz finds evidence for following, but not leading.} Partisans will engage in motivated moral reasoning on many, if not most, political issues \cite{Dittoetal2009}. Similar to how partisan bias warps retrospective evaluations \cite{Bartels2002b}, partisan bias is likely to distort moral judgments.

Theories of how moral judgments affect politics need to account for the dynamics of party identification. Scholars have moved away from the Downsian concept of party identification as a reflection of individuals' agreement with the policy positions of the parties \cite{Downs1957}. Instead, party identification resembles religious affiliation in that it is defined through social affiliations that define good and bad \cite{GreenPalmquistSchickler2002}. Despite recent headlines to the contrary,\footnote{In late 2013 Gallup found that more Americans identified as independents than partisans. Forty-two percent are independents; 31\% are Democrats; and 25\% are Republicans. However, a majority of Americans are still outright partisans (56\%), and fully 32\% of Americans are leaners, leaving only 10\% as pure independents. Source: Jones, Jeffrey M. 2014. ``Record-High 42 percent of Americans Identify as Independents.'' \emph{Gallup.com}. 14 January 2014. \url{http://www.gallup.com/poll/166763/record-high-americans-identify-independents.aspx}. Accessed 15 January 2014.} most Americans are at least marginal partisans, and marginal partisans (leaners) are often just as loyal and committed to the party at the ballot box as stronger identifiers \cite{Keithetal1992}. There is very little partisan ambiguity, but there is evidence of significant issue-oriented ambiguity among the public \cite{FeldmanZaller1992}. Therefore, theories of morality in politics need to incorporate the motivational influence that partisanship has on political judgment.

\citeasnoun{Downs1957}, however, makes a still relevant observation: politics is too distant from citizens' lives to warrant dedicating time and effort toward learning about the issues. Instead, citizens will rely on cues from trusted political elites. Many political issues are too complex to activate the appropriate moral intuitions at a glance. Debates will elicit numerous competing moral considerations. Citizens need elite cues to emphasize the ``right'' moral considerations to emphasize and downplay the competing moral claims. For example, the Affordable Care Act increases health insurance coverage, but it also undeniably increases health care costs for people who previously had plans that are no longer sufficient under the law. The harm/care moral intuition could influence conclusions in either direction.

The dominant theories of morality and politics assumes that partisans and ideological identities result from morals; they largely ignore the impact that these identities can have on morals \cite{Bartels2002b,Goren2005,LaymanCarsey2002b}. \citeasnoun{Dittoetal2009} catalogue a number of ways that moral judgments are susceptible to motivational biases. Evidence from the famous trolley problem suggest that people will change their moral judgments depending on the social groups of those harmed \cite{Uhlmannetal2009}. Moral judgments are susceptible to social bias. Some partisans will change their identities to fit their issue positions, which conforms with the expectations of MFT and MCT, but others will change their positions to fit their party \cite{LaymanCarsey2002b}. Parties convert their identifiers to their positions \cite{Levendusky2009b}, and conflicts between party and ideology ``tend to be resolved in favor of the party'' \cite{Johnstonetal2004}.\footnote{Cited in \citeasnoun[337]{Johnston2006}.} People will ``project'' positions onto candidates in order to minimize distance with preferred candidates and to maximize the distance with disfavored candidates \cite{MarkusConverse1979} (See also \cite{Grahametal2012}). All of these findings of partisan motivational bias suggest that actors are an important consideration when making judgments about the political landscape, especially among partisans who have less information about politics \cite{BradySniderman1985}.

The two criticisms highlight the need to account for perceptions of actors in theories of morals and politics. Some work is being down toward this end (see \citeasnoun{Clifford2014}), but existing theories are too dependent on political issue preferences or information about policy positions to give a complete picture of the role morality plays in political behavior. For example, \citeasnoun{Clifford2014} finds that learning an out-party politician holds a position that is contrary to partisan expectations increases positive character judgments on traits that are connected with individuals' moral foundations. Clifford proposes a theory of moral exemplification where actors' policy positions communicate moral meaning on individuals' moral intuitions. Citizens who oppose the death penalty will increase their positive judgments of how caring politicians who agree are. However, on-line theories of information processing suggest the reasons behind these judgments will be forgotten over time \cite{LodgeMcGrawStroh1989,LodgeSteenbergenBrau1995}. As the reasons for the judgments decay, the judgments of actors will persist. Actors will then be associated with morally-charged emotion and judgments, and they can act similarly to moral intuitions or convictions.

Psychologists have found that political actors have moral meaning. For example, \citeasnoun{Effronetal2009} found that white respondents who expressed support for President Obama then felt they had the moral credentials to favor their race over African Americans. Similarly, other scholars found that President Obama's election in 2008 immediately made whites less supportive of efforts to ameliorate racial disparities \cite{Kaiseretal2009}. The election of an African American president, in a sense, absolved whites of their responsibilities on the moral question of racial justice. Similarly, subconscious primes of attitudes toward politicians can significantly impact political behavior \cite{WeinbergerWesten2008}. Experiments have also shown that the individuals involved in moral problems can affect whether people will rely on deontological or consequentialist moral reasoning \cite{Edmonds2013,Uhlmannetal2009}.

This evidence suggests that moral judgments of actors can compliment theories focused on issues. Issues are not irrelevant to political behavior. They motivate many citizens' political engagement and beliefs \cite{Ansolabehereetal2008}. Nevertheless, an exclusive emphasis on issues neglects the decades of findings pointing to the limited political knowledge \cite{DelliCarpiniKeeter1996} and competence of the American public. Actors matter as moral entities.


%
%\begin{singlespace}
%
%\bibliographystyle{C:/Users/Public/Bibliography/apsr_fs}
%\bibliography{C:/Users/Public/Bibliography/DissertationMasterBibliography}
%
%\end{singlespace}
%
%\end{document} 