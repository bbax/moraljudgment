\chapter{Conclusion: Moralized Actors and American Democracy}

\section{Summary of Findings}

\section{Implications for American Democracy}
The lesson of many of these findings is not particularly optimistic for modern presidential campaigns. The experimental results in Chapter \ref{ch:vignettes} show that attacking opponents' morality is a winning proposition. If a campaign can successfully define the opponent as immoral, it gains increase loyalty from its base and offers the opportunity to win independents. Defining competence does not yield similar rewards, and defining opponent on both morality and competence does not lead to greater support than defining morality. These results suggest that campaigns should hit opponents early and often on their moral character. Running campaigns focused on competence is not as effective as attacking morality --- as Dukakis in '88, Gore in '00, and Romney in '12 learned. Competence is theorized to be more rational \cite{Popkin1991}, but the emotional resonance of morality is more influential. 

Partisans justify their partisanship through a sense that they are supporting the moral candidate, as I found in Chapter \ref{ch:anes}.

\section{What's Next?} 