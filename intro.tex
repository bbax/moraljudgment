%COMMENT OUT DOCUMENT HEADER AND BIBLIOGRAPHY TO CREATE FULL DISSERTATION
%\documentclass[11pt]{book}
%\usepackage{multirow}
%\usepackage{harvard}
%\usepackage{setspace}
%\usepackage{subfig}
%\usepackage{geometry}
%\usepackage{fancyhdr}
%\usepackage{amsmath,amsthm,amssymb}
%\usepackage{graphicx}
%\usepackage{hyperref}
%\usepackage{pdflscape}
%\usepackage{fullpage}
%\usepackage{dcolumn}
%
%
%\linespread{1.6}
%\graphicspath{ {c:/Users/Public/Figures/} }
%
%\begin{document}

\chapter{Introduction: Moral Judgments of Political Leaders}\label{ch:intro}
Very negative attitudes toward politicians are common in American politics. Americans seem to love to hate their political leaders. T-shirts, signs, and bumper stickers maligning presidents and candidates are common. During the presidency of George W. Bush a popular tag line read, ``Somewhere in Texas a village is missing its idiot.'' A man wore a T-shirt at the 2013 national conference of the National Rifle Association that read, ``Obama is a Socialist, a Fraud, a Liar, \& a Racist Bigot.''\footnote{Stopera, Matt. 2013. ``54 Things I Learned at the Biggest Gun Show in the World.'' BuzzFeed.com. Posted 6 May 2013. At http://www.buzzfeed.com/mjs538/things-i-learned-at-the-biggest-gun-show-in-the-world. Accessed 6 May 2013.} The NRA's convention also featured a number of novelty shooting targets that closely resembled President Obama as a zombie.

Disliking political figures is normal, even expected, especially among members of the opposition. Few Republicans approve of the job performance of Democratic presidents, and vice versa. The gap between Republicans' and Democrats' approval of President Obama's job performance was relatively large throughout 2013, but there has been a gap between partisans' job approval ratings since data collection began in the early 1950s.\footnote{Jones, Jeffrey M. 2014. ``Obama's Fifth Year Job Approval Rating Among Most Polarized.'' Gallup Politics. Posted 23 January 2014. http://www.gallup.com/poll/167006/obama-fifth-year-job-approval-ratings-among-polarized.aspx. Accessed 30 January 2014.} The opposition rarely gives the president high job approval, even when he or she performs well. President Clinton oversaw eight years of economic expansion, yet only 29\% of Republicans approved of his job performance when he left office.\footnote{Data on presidential job approval comes from Gallup President Job Approval Center. http://www.gallup.com/poll/124922/Presidential-Approval-Center.aspx.} Democrats were similarly disapproving of President Reagan when he retired.\footnote{Twenty-eight percent of Democrats approved of President Reagan's job performance at the end of his second term.} These kinds of disagreements are a consistent feature of American public opinion. Despite James Madison's famous desire to keep the American Republic free from the forces of faction, American partisans have disagreed about political leaders' job performance since the beginning. A healthy democracy depends on opposition to those in power. Democracy needs people who disagree.

Many people, however, seem to go beyond simply disliking political leaders from the other party; their attitudes seem to surpass dislike and cross into moral condemnation. Their attitudes suggest they believe the leaders of the other party are fundamentally bad and immoral. Even years later, feelings toward political leaders can be very raw and emotional. When former British Prime Minister Margaret Thatcher passed away in 2013, some of her political opponents made the song ``Ding Dong! The Witch is Dead'' a top-selling single to mark the occasion.\footnote{O'Carroll, Lisa. 2013. ``Thatcher's death prompts chart success for Ding Dong! The Witch Is Dead.'' TheGuardian.co.uk. Posted 10 April 2013. At http://www.guardian.co.uk/media/2013/apr/10/thatcher-death-ding-dong-witch. Accessed 30 May 2013.} Protestors at her funeral turned their backs to her coffin. In 2012, some liberal Democratic activists visiting the White House felt so passionate a dislike for Ronald Reagan that they posted pictures online of themselves giving his official portrait the finger.\footnote{Fiorillo, Victor. 2012. ``What Happens When You Let Gay Philly Activists Into the White House?
They Pose for Pics Giving Ronald Reagan?s Portrait the Finger.'' \emph{Philadelphia Magazine}. Posted 22 June 2012. At http://www.phillymag.com/news/2012/06/22/gay-activists-give-ronald-reagan-finger-white-house/. Accessed 30 January 2014.} Even years after these politicians left office, some of their political opponents still feel very negative moral condemnations of these politicians. Their actions suggest they believe Thatcher and Reagan were truly morally bad.

These negative attitudes are not limited to just rank-and-file partisans. The Internet is filled with vitriolic attacks on politicians, but sitting members of Congress get involved too. Republican Congressman Joe Wilson from South Carolina famously shouted ``You lie!'' at President Obama during an address to a joint session of Congress in 2009. Representative Wilson's outburst was a serious violation of professional decorum, and it seemed to originate from a visceral rejection of President Obama as a person and a president. Some people argued the president's race contributed to the bad behavior. Democrats have also expressed their ardent disapproval for a president from the GOP on the floors of Congress. Democratic members of Congress booed President Bush twice during his 2005 State of the Union address.\footnote{Smallen, Jill and Charlie Mitchell. 2005. ``The Week on the Hill, January 31-February 3.'' \emph{National Journal}. Posted 5 February 2005. http://www.nationaljournal.com/member/magazine/the-week-on-the-hill-january-31-february-3-20050205. Accessed 30 January 2014.} Rhetoric from elected members officials is sometimes as heated and negative as what the Internet produces. A Republican member of Congress sent a tweet calling President Obama a ``socialist dictator'' and a liar from the floor of the House before the 2014 State of the Union.\footnote{Blake, Aaron. 2014. ``GOP Congressman Calls Obama a `Socialist dictator.''' \emph{The Washington Post}. Posted 28 January 2014. At http://www.washingtonpost.com/blogs/post-politics/wp/2014/01/28/gop-congressman-calls-obama-a-socialist-dictator/. Accessed 30 January 2014.} Another Republican walked out on the President mid-speech, saying it was a protest of his breaking the oath of office and committing offenses worthy of impeachment.\footnote{Gillman, Todd J. 2014. ``Stockman Walks Out on Obama.'' \emph{DallasNews.com}. Posted 28 January 2014. At http://trailblazersblog.dallasnews.com/2014/01/stockman-walks-out-on-obama.html. Accessed 30 January 2014.} Highly moralized conflicts

It can be tempting to dismiss these attitudes as political posturing or the acts of extremists, but very negative opinions of elected leaders have real political consequences. The strongest predictor of identifying with the Tea Party is believing that President Obama is a socialist \cite{ParkerBarreto2013}. There are other contributors to identifying with the Tea Party, but none has a relationship as strong as negative beliefs about President Obama. Sixty to 70 percent of Republicans support the Tea Party; it is not a fringe movement, despite what many commentators say.\footnote{Fuller, James. 2014. ``What shutdown? New poll shows tea party support remains steady.'' \emph{WashingtonPost.com}. Posted 28 January 2014. http://www.washingtonpost.com/blogs/the-fix/wp/2014/01/28/what-shutdown-new-poll-shows-tea-party-support-remains-steady/. Accessed 30 January 2014.} The Tea Party has greatly affected U.S. politics since 2009, especially in the U.S. House of Representatives and in state capitals around the country. The fundraising efforts of the Netroots movement that emerged to oppose President Bush contributed to the Democrats' retaking Congress in 2006 \cite{Karpf2013}.

A large body of research over the past two decades has found that morals can greatly impact political behavior. This existing research focuses primarily on moral issue attitudes \cite{Haidt2012}, policies \cite{Meier1994,MooneySchuldt2008}, and rhetoric \cite{Marietta2009,Grahametal2009}. The research only fleetingly examines how morals relate to the men and women who pursue and hold elected office. Moralized attitudes toward political actors powerfully influence how morals and politics interact. Data-driven studies of attitudes toward political actors have largely ignored their moral dimension. Research in social psychology has identified morality, or warmth, as a key dimension of social perception \cite{CuddyFiskeGlick2007}. However, political scientists have not incorporate this understanding into theories of political evaluation. There is a significant body of qualitative and historical work that examines the top-down implications and origins of presidents' and other leaders' moral status and leadership \cite{DaynesSussman2001,Denton2005,Hinckley1990,Kane2001,Olasky1999,Pfiffner2004}. There are few studies that explicitly address the bottom-up moral perceptions of the mass public. Scholars and presidents believe that the executive is a source of moral leadership and moral symbolism, yet we know very little about citizens' moral judgments of major political actors in light of new theories of morality. We know even less of how those judgments affect important political behaviors.

Scholars have examined for decades how the public judges political actors, especially the president, yet they have largely ignored the moral content of attitudes toward these actors. Studies have split between studying the emotions that these actors elicit \cite{HuddyFeldmanCassese2007,Marcus2000,Petersen2010} and citizens' character judgments of these leaders \cite{Funk1999,Goren2002,Goren2007,Hayes2005,Kinderetal1980,Newman2003,Newman2004}. The literature on perceptions of political actors treats these two domains as distinct, but recent developments in the study of morals and politics suggest that the two may be intimately connected. \citeasnoun{Haidt2001} argues that moral judgments are largely automatic and rooted in affective moral intuitions. Intuitions are emotionally charge cognitions, but they are not a form of reasoning \cite[814]{Haidt2001}. Many stimuli can automatically activate these intuitions and their impact is often entirely subconscious. These arguments suggest that character judgments may be closely related to moral feelings and intuitions. Political actors, especially the president, are often a central stimuli available to citizens for forming opinions. Politicians from the other party can be highly influential cues that activate strong emotions and judgments \cite{Nicholson2012}, which suggests that citizens may react to actors similarly to how they react to other moral entities.

I argue that explanations of how morals impact politics need to take into account moral judgments of actors. Existing theories tend to focus on moral issues and policies, typically abortion or gay marriage, and ignore actors' moral meaning. It is likely that citizens have moral attitudes toward actors themselves. Issues will influence these judgments, but I argue that moral judgments of political actors have an independent impact on political behavior. As citizens learn more about politicians they oppose they develop affectively charged, values-based character judgments of those actors. These judgments are moral.

When individuals encounter information about actors that activates moral intuitions and convictions, the moral emotions and judgments transfer to the actor \cite{LodgeSteenbergenBrau1995}.\footnote{My theory is agnostic as to whether the information deals with the actors' character, personality, issue positions, or behavior. So long as the information serves to activate moral judgments and intuitions, it does not matter what type of information it is.} The emotional impact of this information remains even if details do not.This process turns politicians into morally meaningful entities. The right has vilified President Obama in moral terms. President Obama himself, then, has moral meaning for many Republicans. A similar process occurred among Democrats regarding George W. Bush. Recent work has shown that affective polarization is increasing \cite{Iyengaretal2012}; partisans now feel much more hostility toward the out-party than in decades past. I argue that one neglected explanation for partisan hostility may be that partisans perceive the other party's leaders as morally bad, not just ideologically misguided or politically incompetent.

Partisan identification is a key element of these judgments. Americans who identify with a political party are most likely to morally judge politicians. Such judgments are much less likely to develop among pure independents. Partisanship is an important determinant of how individuals see and interpret politics \cite{Bartels2002b,GreenPalmquistSchickler2002,Rahn1993}. Partisanship moderates and mediates almost everything people encounter. Many Republicans condemn almost anything President Obama says or does because to them he is a morally negative entity. Similarly few Democrats believe President Obama is Marxist intent on destroying America, but many did worry President George W. Bush was a fascist. These kinds of negative moral judgments are almost exclusively reserved for the leaders of the other party.

In the rest of the chapter, I lay out an argument for combining moral context with partisanship into a theory of moral judgments of political actors. Drawing on recent developments from experimental philosophy, neuroscience, social psychology, and political science, I argue that moral judgments of political actors are an important determinant of Americans' political behavior. These judgments go beyond simply disliking a politician to true moral condemnation.\footnote{I focus on moral condemnation and leave aside moral praise for the time being. Partisans are enthusiastic about their party's presidential candidates during campaigns, but enthusiasm seems to fade. Democrats have been much less enthusiastic about President Obama since 2008 than the Tea Party has been in condemning him. Future research will have to address positive moral evaluations, though my argument largely applies to both positive and negative moral judgments.}




%\section{The \emph{Political} Trolley Problem}
%Most theories of morals and politics from social psychology are agnostic about the institutional environment where the two meet. \citeasnoun{Haidt2012} mentions that the moral intuitions he describes have been found in many cultures, but he does not explore how different political institutions (political parties, forms of government, election rules, etc.) would impact how his theory of moral foundations would impact actual political behavior. Examining the many ways that morals and political institutions interact across the world's regimes is beyond the scope here, but it is worth looking more closely at how they interact in the American context where the theories were formulated.
%
%With some modification the famous trolley-problem thought experiments can illustrate why institutions and actors need consideration when studying morals and politics. In its basic form, the trolley problem explores whether it is permissible to intentionally kill one person trapped before a runaway trolley if it saves five lives on another track \cite{Foot1967}. The moral decision is whether a bystander should flip a switch and divert the train. Foot believed diverting the trolley to kill one so that five may live---even if the one fatality is intentional---was the moral decision. For centuries, the prevalent doctrine of double effect formulated by Aquinas stipulated that any intentional action to harm another is morally wrong, but actions that are not intentionally harmful which have harmful effects can be morally right \cite{Edmonds2013}. A second variation involved pushing a large person from a footbridge overhead into the path of the trolley to save five people stuck further down \cite{Thomson1985}.
%
%There exist dozens of variations on the trolley problem that explore differences in how people make moral judgments. Experimental philosophers, neuroscientists, and psychologists have found that people make very different moral judgments based on the contexts of those judgments. The first variant, involving pulling a lever, reveals that most people will rely on consequentialist moral reasoning when acting impersonally and from a distance; however, the second variant, footbridge, has shown that most people employ non-negotiable, deontological reasoning when they need to act intimately and personally. They conclude it is wrong to harm others, even if the benefits are rationally greater. Researchers have also found that the identity of the person who will be hurt matters. On the footbridge, liberals are less likely to push individuals with non-white names; conservatives are indifferent [CITATION?].
%
%The trolley problem and similar research has yielded great insights into how humans make moral judgments; however, there are four central shortcomings for applying these lessons to political life. The first is that in politics individuals very rarely, if ever, judge an act they will make themselves. They are judging the morality of politicians' actions and agenda. Second, the actors that citizens judge have partisan affiliations. The belong to one ``team'' or the other. Third, institutions, especially delegation, shape the appearance and attribution of blame. Decision-makers rarely carry out the morally-charged actions they determine are necessary. They make decisions, and others carry them out. Fourth, there are other actors who actively justify or condemn the actions of the men and women turning the lever or pushing the fat man. Together these shortcomings suggest that political institutions and moral reasoning will make moral judgments of political actors as important as judgments of political actors.
%
%First, much of the existing research into moral reasoning studies judgments for actions that might plausibly happen in any individual's life. We can all imagine ourselves in a situation where we might need to make a decision similar to those of the trolley problem. But in politics, the vast majority of citizens are judging the morality of others' actions and decisions. For most citizens in most situations, political moral judgments are impersonal and distant. Furthermore, most political judgments regard actions ordinary citizens will never have to take. The vast majority of Americans will never have to decide whether to invade Iraq, extend voting rights to the disenfranchised, or establish regulation for marriage and abortion. These are exactly the judgments that citizens must make when they encounter morally relevant information about politics.
%
%%To highlight how this might affect political moral judgments, we can modify the trolley problem with the addition of a second actor called the ``station master'' The station master is positioned at the lever as the trolley barrels out of control, and he or she is in position to push the man. Observers must make many assumptions and conjectures regarding the station master's state of mind, intentions, etc. when he or she makes a moral decision. Furthermore, the station master can justify his or her decision and try to persuade citizens to agree. Citizens must decide whether to trust these justifications, among many other considerations. The station master is an elected official, such as the president, who has the implicit support of the public. The judgment in these cases are not spur of the moment; rather they are secondary judgments of the justifications the station master offers. This situation would suggest that reflection and reasoning might be more influential; however, that suggestion ignores the limited attention and time people give to politics \cite{Downs1957,Prior2007} and the limited knowledge they possess about political issues \cite{DelliCarpiniKeeter1996}.
%
%In politics, actors are also not neutral actors; they are affiliated with group that many citizens, especially the most politically active, hold strong, influential predispositions toward: political parties.\footnote{There are other institutional factors likely to influence these judgments as well, such as federalism, holding the office vs. seeking it, etc.} The type of reasoning individuals dedicate in these cases will depend on partisanship. If the station master is a member of the same party, individuals will dedicate more effort to justifying the decision he or she made \cite{Petersenetal2013}. On the other hand, if the station master is a member of the other party, quick negative affective moral reactions may be more influential.  Instead of thinking up a justification for why ``their'' station master was right, they will \emph{feel} that the ``other'' station master was wrong. The station master will be a prototypical representation what they are not \cite{Nicholson2012}, which is an easier, less demanding judgment to make. Partisans will construct motivated moral judgments to support their party or criticize the other \cite{TaberLodgeGlathar2001}. Sometimes it seems criticism is less cognitively demanding than justification. For example, in the debate surrounding the Affordable Care Act, most rank-and-file Republicans (even many elected Republicans) have simply and repeatedly asserted that ``Obamacare'' is wrong, period, without comparing its strengths and weaknesses to an alternative. Regular Republicans cannot explain exactly why they think it is wrong---at least initially---but they know it is wrong because of its association with the morally bad President Obama.
%
%We can further modify the trolley set-ups by replacing the station master with an ``assistant station master.'' The station master still makes the decision to pull the lever or push the man, but he or she does not take the action. That responsibility is delegated to the assistant. Physical distance from a moral action impacts how people judge its morality \cite{Dittoetal2009,Pahariaetal2009}. Distance can complicate the judgment, but it can also simplify it because partisans attribute blame without working out who is really at fault. People tend to blame or credit politicians for a wide range of events that are outside their control \cite{AchenBartels2006,HealyMalhotraMo2010}. Again, the partisan affiliation of the station master will influence how individuals morally assess the situation. For example, when the U.S. occupation of Iraq faltered, partisanship influenced who Americans blamed. Democrats tended to blame President Bush; while Republicans gave the blame to others.\footnote{Fifty-four percent of Democrats blamed the White House for the current situation in Iraq in 2007; only 13\% of Republicans blamed the administration. Republicans were more likely to blame either Congress (20\% of Republicans vs. 7\% of Democrats) or the Iraqi leadership (52\% vs. 25\%). Independents tended to blame the White House (46\%), though they blamed Iraqi leaders more than Democrats (29\%). Princeton Survey Research Associates International/Newsweek Poll, July 11-12, 2007. Retrieved Jan-31-2014 from the iPOLL Databank, The Roper Center for Public Opinion Research, University of Connecticut. \url{http://www.ropercenter.uconn.edu/data_access/ipoll/ipoll.html}.} The structure of the federal government may also affect how moral judgments are made. There is not a single station master in the U.S. government; there are three who jostle and maneuver to make moral decisions. The president is dominant in the public eye, but he often cannot act alone.
%
%The last modification is the addition of outside ``lawyers'' who advocate for or against the station master's decision. These lawyers are the media, campaign communications, conversations with opinionated friends, Facebook posts---any entity that provides political information. People rarely observe politics firsthand. They rely on the lawyers for information. Some lawyers aim for neutrality, but many do not.\footnote{And many who try for neutrality arguable fail.} The lawyers try to justify or condemn the station master's decision. They have their own biases and motivations for doing so. Some are employees of the station master or the master's opponents. Some are independent organizations with their own agenda. These actors use moral rhetoric to shape the public's opinions \cite{CliffordJerit2013,Grahametal2009}, and they are a central component of moral judgments in politics because people must rely on them for their political information. Furthermore, many of these lawyers focus their attacks on the actor as a person. For example, media coverage of campaigns used to emphasize the parties nearly as much as individual personalities, but by the late 1990s most campaign coverage focused on the candidates. For every time the media mentioned the parties in 1952, they mentioned the candidates 1.7 times. That ratio increased to 5.6 mentions of the candidate for every one of the parties by 1996 \cite{McAllister2007}.\footnote{This increase occurred in other Western Democracies as well, though the U.S. shift was among the largest.} Similarly, even when they discuss policy positions, campaigns' attack advertising tends to focus on the candidates themselves \cite{Geer2006}. Though attack ads are less focused on personal traits than is often thought, many are still focused on the candidates instead of the issues \cite{Geer2006}.
%
%The additive effect of these modifications creates a situation where the cognitive strategies people use are the product of two contextual variables: the moral context (e.g. footbridge vs. lever) \emph{and} the partisan context. The features of my political trolley problems all point toward partisanship being a key variable for understanding citizens' moral judgments and attitudes.


\section{Moral Judgment without the Scandal}
Benghazi as confirmation for Republicans' pre-existing beliefs that President Obama was an immoral person who would negligently allow four Americans to die and then spin it for political points. In situations like this, a scandal is not truly exogenous. Citizens' reactions to the new information is not truly external to their partisanship and previous judgments of the actor.


%\section{When Politics Gets Personal}
%We can think about a judgment of a political actor comprising two primary dimensions: policy and personal. Table \ref{tab:actor_dimensions}

%\begin{table}[htb]
%\centering
%\caption[Dimensions of Actor Judgments]{\textbf{Dimensions of Actor Judgments}}\label{tab:actor_dimensions}
%\begin{tabular}{|l|l|c|c|}\hline
%\multicolumn{2}{|c|}{} & \multicolumn{2}{c|}{\textbf{Policy}} \\ \cline{3-4}
%\multicolumn{2}{|c|}{} & Like & Dislike \\ \hline
%\multirow{2}{*}{\textbf{Personal}} & Like & & \\ \cline{2-4}
%& Dislike & & \\ \hline
%\end{tabular}
%\end{table}
%
%
%
%\begin{table}[htb]
%\centering
%\caption[Dimensions of Personal Judgments]{\textbf{Dimensions of Personal Judgments}}\label{tab:personal_dimensions}
%\begin{tabular}{|l|l|c|c|}\hline
%\multicolumn{2}{|c|}{} & \multicolumn{2}{c|}{\textbf{Competence}} \\ \cline{3-4}
%\multicolumn{2}{|c|}{} & Competent & Incompetent \\ \hline
%\multirow{2}{*}{\textbf{Morality}} & Moral & & \\ \cline{2-4}
%& Immoral & & \\ \hline
%\end{tabular}
%\end{table}

%
%\begin{singlespace}
%
%\bibliographystyle{C:/Users/Public/Bibliography/apsr_fs}
%\bibliography{C:/Users/Public/Bibliography/DissertationMasterBibliography}
%
%\end{singlespace}
%
%\end{document} 