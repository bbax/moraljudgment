This dissertation examines how citizens form and then act on moral judgments of political leaders at the presidential level in the United States. Recent work connect morality with political behavior has under-emphasized the impact that judgments of political actors. To overcome this deficit in the literature, I argue that moral judgments of the candidates are an important means for citizens to encode their moral reactions to politics. 

I build a mediation model for how moral judgments of actors will impact voting. Citizens store moral emotions and judgments in their judgments of political leaders' morality. The reasons for these judgments are forgotten, but their impacts are preserved through the moral judgments individuals make of leaders. After these moral impressions are formed but the reasons forgotten, the actors themselves can become an important, morally-significant intuition that affects citizens engagement with politics.

I find that judgments of candidates' moral character are built on a foundation of moral emotions, especially anger and disgust. Negative moral judgments of out-party leaders make citizens feel aversion rather than anxiety, and aversion is particularly morally relevant. Moral judgments have different emotional foundations than judgments of competence.

I show experimentally that moral judgments of actors have a larger effect on vote choice than competence. Participants primed to believe that an out-party president is not moral are significantly more likely to vote against that president. They are also more likely to support down-ballot candidates who oppose the immoral president. Furthermore, combining negative primes of morality and competence do not have a stronger effect than negative primes regarding morality.

Additional experimental evidence from a conjoint design reveals that partisanship does not overwhelm the impacts of moral character. Respondents do not rely on partisanship at the expense of all character considerations. The results suggest that individuals may form moral judgments through different process, but once they are formed they take a rely on a variety of moral judgments to broadly identify and support the most moral candidate for president.

Observational evidence from the American National Election Studies from 1984 to 2012 shows that partisans who are ambivalent or negative between their party's presidential nominee compared to the other party's candidate are significantly less likely to be loyal at the ballot box. This effect is often significantly stronger than the effect of issue attitudes. Moral judgments of the candidates appears to be an important pathway for reasoning through partisans' vote choice.

A panel study from the 2008 National Annenberg Election Study shows that partisan voters who change their moral judgments of the two candidates to favor their own party are also more likely to change their vote to support their party's nominee. 