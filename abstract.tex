This dissertation examines how citizens form and then act on moral judgments of political leaders, especially the President of the United States. Popular narratives of elections emphasize candidates; every cycle spawns dozes of op-eds and blog posts focused on how candidates matter. Political scientists, on the other hand, have largely found that candidates are not central to citizens' vote choices. These findings are the result of looking at candidate judgments across parties instead of within them. Furthermore, recent work connect morality with political behavior has focused on the role of political issues, which under-emphasizes the impact that judgments of political actors can have.

The dissertation builds a theory for how moral judgments of actors will impact political behavior with an emphasis on vote choice. Citizens store previous emotional and other impressions of actors in their judgments of political leaders' morality. The reasons for these judgments are typically forgotten, but their impacts are preserved through on-line information processing. After these moral impressions are formed but the reasons forgotten, the actors themselves can become an important and morally significant signal for how citizens engage with politics.

I find that moral trait judgments are connected to strong moral emotions, especially anger and disgust. Negative moral judgments of out-party leaders make citizens feel more aversion than anxiety. Moral judgments have different emotional foundations than judgments of competence, and these differences give moral judgments unique and powerful impacts on political behavior.

I show experimentally that moral judgments of actors have a larger effect on vote choice than competence. Participants primed to believe that an out-party president is not moral are significantly more likely to be confident that they will cast a vote against that president. Furthermore, combining negative primes of morality and competence do not have a stronger effect than negative primes regarding morality.

Additional experimental evidence from a conjoint design reveals that partisanship does not overwhelm the impacts of moral character. Respondents do not rely on partisanship at the expense of all character considerations. The results suggest that individuals may form moral judgments through different process, but once they are formed they take a rely on a variety of moral judgments to broadly identify and support the most moral candidate for president.

Observational evidence from the American National Election Studies between 1984 and 2012 shows that partisans who are ambivalent or negative between their party's presidential nominee compared to the other party's candidate are significantly less likely to be loyal at the ballot box. This effect is often significantly stronger than the effect of issue attitudes. Moral judgments of the candidates appears to be an important pathway for reasoning through partisans' vote choice.

A panel study from the 2008 National Annenberg Election Study shows that partisan voters who change their moral judgments of the two candidates to favor their own party are also more likely to change their vote to support their party's nominee. 